\documentclass[a4paper]{article}
\usepackage[english]{babel}
\usepackage{multicol}
\usepackage[utf8]{inputenc}
\usepackage{amsmath}
\usepackage{graphicx}
\usepackage{amssymb}
\usepackage{enumitem}

\usepackage[colorinlistoftodos]{todonotes}

\title{Principle of Data Base Systems : Assignment 1}
\date{\today}
\author{Lucky Sahani}

\begin{document}
\maketitle

\section{Question 2}
\label{sec:q1}
{\bf Question :} Identify dependencies (Trivial, Fully Functional, Transitive, Multi-valued) in the context of  academic system at IIT Kanpur. 
\subsection{Trivial Dependencies}
A trivial functional dependency occurs when you describe a functional dependency of an attribute on a collection of attributes that includes the original attribute i.e it is only a trivial functional dependency when a subset of attributes depends from the full set. \\Definition of trivial functional dependency: Example.$: (a,b)$ depends from $(a,b,c)$ attributes.
\begin{itemize}
\item (Name, Roll Number) -$>$ Name
\item (Course Name, Course Code) -$>$ Course Name
\item (Roll No., CPI) -$>$ CPI
\item (Name , Date of Birth) -$>$ Date of Birth
\item (Roll No., Email) -$>$ Email
\end{itemize}

\subsection{Fully Functional Dependencies}
A full functional dependency occurs when you already meet the requirements for a functional dependency and the set of attributes on the left side of the functional dependency statement cannot be reduced any farther.
\begin{itemize}
\item Course Number -$>$ (Course Name, Instructor, Credits, Prerequisite)
\item Department Name -$>$ (Department Code, Department Area)
\item Lab Course No. -$>$ (Lab Location, Lab Department)
\item (Course No., Day) -$>$ (Lecture Time, Lecture Location)
\item Roll No. -$>$ (Name, Email, Address, CPI, Department, Sex)
\end{itemize}

\subsection{Transitive Dependencies}
Transitive dependencies occur when there is an indirect relationship that causes a functional dependency.
\begin{itemize}
\item (Course, Roll No.) -$>$ Grade -$>$ (Pass/Fail/Not Done)
\item (Roll No., Semester) -$>$ (Credits) -$>$(AP/not)
\item Roll No. -$>$ Address -$>$ Hall No.
\item Email -$>$ Roll No. -$>$ Name
\item Roll No. -$>$ Hall -$>$ Warden-In-Charge
\end{itemize}

\subsection{Multi valued Dependencies}
Multivalued dependencies occur when the presence of one or more rows in a table implies the presence of one or more other rows in that same table. 
\begin{itemize}
\item Course No. -$>$ Instructor
\item Course No. -$>$ Reference Books
\item Course, Book and Instructor : (Course-$>$Book) and (Course-$>$Lecturer)
\item (Name, Address, Phones, Subjects Liked) :A Student’s phones are independent of
the subject they like and each of a student’s phones appears with each of the subjects they like in all combinations
\item (Course No., Labs No., Instructor) : There can be many labs and many instructor for a instructor.Hence, Redundant Information.
\item (Course No, Student Roll No., Instructor) :Same as above

\end{itemize}

\end{document}